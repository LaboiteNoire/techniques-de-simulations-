\documentclass[10pt,a4paper]{article}
\title{Techniques de Simulations}
\author{
  PRISCILLA, GOGUY\\
  \texttt{priscilla.goguy@etu.univ-lyon1.fr}
  \and
  MAHLÎ, REINETTE\\
  \texttt{mahli.reinette@etu.univ-lyon1.fr}
}
\date{13/10/2025}
\usepackage[utf8]{inputenc}
\usepackage[T1]{fontenc}
\usepackage[french]{babel}
\usepackage{graphicx}
\usepackage{hyperref}

\usepackage{listings}
\usepackage{xcolor}


\usepackage[many]{tcolorbox}
\usepackage{lipsum}
\usepackage{tikz}
\usetikzlibrary{automata, positioning, arrows}
\usepackage{pgfplots}
\usepackage{xcolor}
\usepackage{amsfonts}
\usepackage{dsfont}
\usepackage{xr-hyper} 
\usepackage{hyperref} 
\externaldocument[B-]{docB}[Annexe01.pdf]% <- full or relative path


% le préambule











% initialisation des couleurs
\definecolor{codegreen}{rgb}{0,0.6,0}
\definecolor{codegray}{rgb}{0.5,0.5,0.5}
\definecolor{codepurple}{rgb}{0.58,0,0.82}
\definecolor{backcolour}{rgb}{0.95,0.95,0.92}






%block de code 
\lstdefinestyle{mystyle}{
    backgroundcolor=\color{backcolour},   
    commentstyle=\color{codegreen},
    keywordstyle=\color{magenta},
    numberstyle=\tiny\color{codegray},
    stringstyle=\color{codepurple},
    basicstyle=\ttfamily\footnotesize,
    breakatwhitespace=false,         
    breaklines=true,                 
    captionpos=b,                    
    keepspaces=true,                 
    numbers=left,                    
    numbersep=5pt,                  
    showspaces=false,                
    showstringspaces=false,
    showtabs=false,                  
    tabsize=2
}

\lstset{style=mystyle}






%initialisation graphics
\tikzset{
->,
node distance = 2cm}



















% le corps du document






% le préambule
\begin{document}
%titre
\pagecolor{black!20}
\maketitle


\begin{center}
\textbf{\textit{M1 Actuariat}}

\textbf{\textit{ISFA}}

\includegraphics[width=4cm,height=2cm]{img1}

\href{https://github.com/LaboiteNoire/techniques-de-simulations-}{\includegraphics[width=1cm,height=1cm]{img2}}

\end{center}


\newpage

\phantom{aaaaaa}

\tableofcontents

\newpage

\part{Préambule}

\subsection{Avant-propos}
\begin{center}
\begin{minipage}[r]{0.1\textwidth}

\end{minipage}
\begin{minipage}[r]{0.8\textwidth}

Toutes les ressources utilisées seront présentes en annexes et sur le \href{https://github.com/LaboiteNoire/techniques-de-simulations-}{\textit{\textbf{repository}}} github ci-dessus.
Dans une optique de rigueur absolue, nous tenterons de commenter chaque partie de notre code et essaierons de justifier nos méthodes de simulation.

\end{minipage}
\end{center}


\subsection{Notations}

\begin{description}

\item[\bsc{* $\mathcal{P}(\textbf{X})$ : }]

\textit{Soit \textbf{$\Omega$}, un ensemble quelconque.}

\phantom{aaaaaaa} — \textit{On note $\mathcal{P}(\Omega)$, l'ensemble des parties de $\Omega$.}

\item[\bsc{* $\sigma$-Algèbre de $\Omega$ : }]

\textit{Soit $\Omega$, un ensemble quelconque. $\mathcal{A} \subseteq \mathcal{P}(\Omega)$ est dite $\sigma$-Algèbre de $\Omega$ si :}

\phantom{aaaaaaa} — \textit{$\Omega$ $\in \mathcal{A}$}

\phantom{aaaaaaa} — \textit{$\forall A \in \mathcal{A}$, $\bar{A} \in \mathcal{A}$}

\phantom{aaaaaaa} — \textit{$\forall (A_n)_{n \in \mathbb{N}} \in \mathcal{A}^{\mathbb{N}}$, $\bigcup_{n \in \mathbb{N}} A_n \in \mathcal{A}$}

\phantom{aaaaaaaaaaaaaaa} — \textit{On notera la « tribu borélienne » : $\mathbb{B}(\mathbb{R}^n)$ et \phantom{aaaaaaaaaaaaaa} $\lambda$ : « mesure de Lebesgue ».}


\item[\bsc{* $\mu$ Mesure de probabilité sur $(\Omega, \mathcal{A})$ : }]

\textit{Soit $(\Omega, \mathcal{A})$, un couple dit « espace probabilisable »}

\phantom{aaaaaaa} — \textit{$\mu : \mathcal{A} \longrightarrow [0,1]$}

\phantom{aaaaaaa} — \textit{$\mu(\emptyset) = 0$}

\phantom{aaaaaaa} — \textit{$\forall (A_n)_{n \in \mathbb{N}} \in \mathcal{A}^{\mathbb{N}}$, une famille disjointe, $\mu(\bigsqcup_{n \in \mathbb{N}} A_n)$ = \phantom{aaaaaaaaaaaaaa} $\sum_{n \in \mathbb{N}} \mu(A_n)$}

\phantom{aaaaaaaaaaaaaaa} — \textit{On notera la « probabilité historique » : $\mathbb{P}$.}

\phantom{aaaaaaaaaaaaaaa} — \textit{On notera $(\Omega, \mathcal{A}, \mu)$ : « espace probabilisé ».}


\item[\bsc{* \textbf{X}, une variable aléatoire sur $(\Omega, \mathcal{A}, \mathbb{P})$ : }]

\textit{Soit $(\Omega, \mathcal{A}, \mathbb{P})$ : « espace probabilisé »}

\phantom{aaaaaaa} — \textit{$(\textbf{E}, \mathcal{E})$ : « espace mesurable »}

\phantom{aaaaaaa} — \textit{\textbf{X : } $(\Omega, \mathcal{A}, \mathbb{P})$ $\longrightarrow$ $(\textbf{E}, \mathcal{E})$}

\phantom{aaaaaaaaaaaaaaa} — \textit{$\forall B \in \mathcal{E}$, $\mathbb{P}_{X}(B)$ = $\mathbb{P}(\{ \omega \in \Omega | X(\omega) \in \mathcal{B} \})$}


\item[\bsc{* \textbf{X}, une variable aléatoire continue sur $(\Omega, \mathcal{A}, \mathbb{P})$ : }]

\textit{Soit $(\Omega, \mathcal{A}, \mathbb{P})$ : « espace probabilisé »}

\phantom{aaaaaaa} — \textit{$(\mathbb{R}^n, \mathbb{B}(\mathbb{R}^n), \lambda)$ : « espace mesuré »}

\phantom{aaaaaaa} — \textit{\textbf{X : } $(\Omega, \mathcal{A}, \mathbb{P})$ $\longrightarrow$ $(\mathbb{R}^n, \mathbb{B}(\mathbb{R}^n), \lambda)$}

\phantom{aaaaaaa} — \textit{$\mathbb{P}_{X}$ << $\lambda$\footnotemark[1]}

\phantom{aaaaaaaaaaaaaaa} — \textit{$\exists ! \rho : (\mathbb{R}^n, \mathbb{B}(\mathbb{R}^n), \lambda) \longrightarrow \mathbb{R}^{+}$\footnotemark[2], mesurable, tel que :}

\[ \mathbb{P}_{X}(A) = \int_{A} \rho(x) \,dx \]

\end{description}


\footnotetext[1]{$\forall A \in \mathbb{B}(\mathbb{R}^n)$, tel que $\lambda(A) = 0 \Longrightarrow \mathbb{P}_{X}(A) = 0$.}

\footnotetext[2]{Dite : \textit{« densité de \textbf{X} ».}}








\begin{description}

\item[\bsc{* Espérance \textbf{E[X]}, de la variable aléatoire \textbf{X} : }]

\textit{Soit \textbf{X}, une variable aléatoire sur $(\Omega, \mathcal{A}, \mathbb{P})$}

\phantom{aaaaaaa} — \textit{On note} \textbf{E[X]}, \textit{« l'espérance de \textbf{X} ».}

\phantom{aaaaaaaaaaaaaaa} — \textit{Si \textbf{X}, est discrète\footnotemark[1], alors :}

\[ \textbf{E[X]} = \sum_{k \in X(\Omega)} k \mathbb{P}_{X}(\{ k \}) \]

\phantom{aaaaaaaaaaaaaaa} — \textit{Si \textbf{X}, est continue de densité $\rho$, alors :}

\[  \textbf{E[X]} = \int x\rho(x) \,dx \]



\item[\bsc{* Fonction de Répartition $F_X$, de la variable aléatoire \textbf{X} : }]

\textit{Soit \textbf{X}, une variable aléatoire sur $(\Omega, \mathcal{A}, \mathbb{P})$}

\phantom{aaaaaaa} — \textit{On note} $\forall x \in \mathbb{R}$, $F_X(x) = \mathbb{P}_{X}(]-\infty, x])$ \textit{« la fonction de \phantom{aaaaaaaaaaaaaaa} répartition de \textbf{X} ».}



\item[\bsc{* Espace $L^p(\Omega, \mathcal{A}, \mathbb{P})$ : }]

\textit{Soit $(\Omega, \mathcal{A}, \mathbb{P})$}

\phantom{aaaaaaa} — \textit{On note} $L^p(\Omega, \mathcal{A}, \mathbb{P})$, \textit{l'ensemble :}

\begin{displaymath}
\{ \textbf{X} \text{variable aléatoire sur } (\Omega, \mathcal{A}, \mathbb{P}) | \textbf{E[}|\textbf{X}|^p\textbf{]} < + \infty \}
\end{displaymath}


\item[\bsc{* Moment d'ordre $p$ $m_p$\textbf{[X]}, de la variable aléatoire \textbf{X} : }]

\textit{$\forall$ \textbf{X}, $\in L^p(\Omega, \mathcal{A}, \mathbb{P})$}

\phantom{aaaaaaa} — \textit{On note} $m_p$\textbf{[X]} = $\textbf{E[}\textbf{X}^p\textbf{]}$


\item[\bsc{* Moment centré d'ordre $p$ $\mu_p$\textbf{[X]}, de la variable aléatoire \textbf{X} : }]

\textit{$\forall$ \textbf{X}, $\in L^p(\Omega, \mathcal{A}, \mathbb{P})$}

\phantom{aaaaaaa} — \textit{On note} $\mu_p$\textbf{[X]} = $\textbf{E[}(\textbf{X} - \textbf{E[X]})^p\textbf{]}$


\item[\bsc{* Variantes du moment centré d'ordre $2$, de la variable aléatoire \textbf{X} : }]

\textit{$\forall$ \textbf{X}, $\in L^2(\Omega, \mathcal{A}, \mathbb{P})$}

\phantom{aaaaaaa} — \textit{On note} \textbf{V}$(X)$ = $\mu_2$\textbf{[X]}, \textit{la : « variance de \textbf{X} »} 

\phantom{aaaaaaa} — \textit{On note $\sigma_{X}$ = $\sqrt{V(X)}$, « l'écart-type de \textbf{X} »} 





\item[\bsc{* Covariance, entre les variables aléatoires \textbf{X} et \textbf{Y} : }]

\textit{$\forall ($\textbf{X}, \textbf{Y}$)$ $\in L^2(\Omega, \mathcal{A}, \mathbb{P})^2$}


\phantom{aaaaaaa} — \textit{On note} \textbf{Cov}$(X,Y)$ = $\textbf{E[} (\textbf{X} - \textbf{E[X]})(\textbf{Y} - \textbf{E[Y]}) \textbf{]} $, \textit{la : « covariance entre \textbf{X} et \textbf{Y} »} 

\phantom{aaaaaaaaaaaaaaa} — \textit{$(L^2(\Omega, \mathcal{A}, \mathbb{P})$, \textbf{Cov}$)$, forme un « espace euclidien »} 



\item[\bsc{* Coefficient de corrélation linéaire, entre les variables aléatoires \textbf{X} et \textbf{Y} : }]

\textit{$\forall ($\textbf{X}, \textbf{Y}$)$ $\in L^2(\Omega, \mathcal{A}, \mathbb{P})^2$}


\phantom{aaaaaaa} — \textit{On note} $\rho_{X,Y}$ = $\dfrac{\textbf{Cov(X,Y)}}{\sigma_{X} \sigma_{Y}}$, \textit{la : « le coefficient de corrélation linéaire entre \textbf{X} et \textbf{Y} »}



\end{description}

\footnotetext[1]{$\mathbb{P}_{X}$ << $\mu$, ou $\mu$ est : la « mesure de comptage ».}



\newpage
\subsection{Objets du Problème}
\begin{center}
\begin{minipage}[r]{0.1\textwidth}

\end{minipage}
\begin{minipage}[r]{0.8\textwidth}

On cherche à approximer la probabilité de ruine pour un modèle de théorie de la ruine en
assurance moto. On étudie deux cas : un cas où les assurés sont indépendants, et un cas où ils
sont corrélés à travers des conditions météo communes.

\end{minipage}
\end{center}



\begin{description}

\item[\bsc{* La variable aléatoire $X_{norm}$ sur $(\mathbb{R},\mathbb{B}(\mathbb{R}))$ : }]

\textit{$\forall (x_0,b,\sigma,\delta) \in \mathbb{R} \times (\mathbb{R}_{+}^{*})^3$}

\phantom{aaaaaaa} — \textit{On note} $X_{norm}$,

\phantom{aaaaaaaaaaaaaaa} — \textit{la variable aléatoire continue de densité : $f_{norm}$ :}

\[ \forall x \in \mathbb{R}, \text{  } f_{norm}(x) = \mathds{1}_{[0,b]}(x) \text{ } exp(-\dfrac{(x-x_0)^2}{2\sigma^2}) \text{ } (1 + cos(2 \pi \dfrac{x - x_0}{\delta} )^2) \]



\item[\bsc{* La variable aléatoire $X_{puissance}$ sur $(\mathbb{R},\mathbb{B}(\mathbb{R}))$ : }]

\textit{$\forall (a, \alpha) \in \mathbb{R}_{+}^{*} \times ]1 : + \infty [$}

\phantom{aaaaaaa} — \textit{On note} $X_{puissance}$,

\phantom{aaaaaaaaaaaaaaa} — \textit{la variable aléatoire continue de densité : $f_{puissance}$\footnotemark[1] :}

\[ \forall x \in \mathbb{R}, \text{  } f_{puissance}(x) = \mathds{1}_{[a,+\infty [}(x) \text{ } x^{-\alpha} \times \dfrac{\alpha - 1}{a^{1-\alpha}} \]



\item[\bsc{* La variable aléatoire $Z$ sur $(\Omega, \mathcal{A}, \mathbb{P})$ : }]

\textit{Soit $Z(\Omega)$ = $\mathbb{N}$ et $(p_n)_{n \in \mathbb{N}} \in [0,1]^{\mathbb{N}}$ et tel que : $\sum_{n \in \mathbb{N}} p_n$ = 1}

\phantom{aaaaaaa} — \textit{On note} $Z$,

\phantom{aaaaaaaaaaaaaaa} — \textit{la variable aléatoire discrète, dont les probabilités \phantom{aaaaaaaaaaaaaaaaaaa} respectives sont ainsi notées :}

\[ \forall n \in \mathbb{N}, \text{  } p_n = \mathbb{P}(Z = n) \]


\phantom{aaaaaaaaaaaaaaaaaa} — \textit{On désignera par la suite sa probabilité de la sorte : $\mathbb{P}_Z$}



\item[\bsc{* La variable aléatoire \textbf{X} sur $(\Omega, \mathcal{A}, \mathbb{P})$ : }]

\textit{Soit : $X_{norm}$, $X_{puissance}$, $Z$, des variables aléatoires mutuellement indépendantes, de lois (et ou de densités) respectives : $f_{norm}$, $f_{puissance}$ et $\mathbb{P}_Z$.}

\phantom{aaaaaaa} — \textit{On note} \textbf{X},

\phantom{aaaaaaaaaaaaaaa} — \textit{la variable aléatoire, définie de la sorte :}

\begin{displaymath}
\textbf{X} = 
\begin{cases}
	X_{norm} \text{ si Z = } 0 \\
	X_{puissance} \text{ si Z = } 1 \\
	Z \text{ sinon}
\end{cases}
\end{displaymath}





\end{description}




\footnotetext[1]{\href{run:./Annexe01.pdf}{This is my link}}


\href{file:ax01.pdf}{File keyword}

\href{run:./Annexe01.pdf}{Run keyword}

\href{run:foo.pdf}{Ouvrir l'annexe}

\href{run:./foo.tex}{test}



\section{Organisation du travail}


\part{Modélisation}
\section{modélisation}
\subsection{simulation de $X_{norm}$}
\subsection{Simulation de $X_{puissance}$}
\begin{displaymath}
\forall (a, \alpha) \in \mathbb{R}_{+}^{*} \times ]1 : + \infty [,
\end{displaymath}

\begin{center}
\begin{minipage}[r]{0.1\textwidth}

\end{minipage}
\begin{minipage}[r]{0.8\textwidth}
$X_{puissance}$ est une variable aléatoire continue, de densité $f_{puissance}$, tel que $\forall x \in \mathbb{R}$ :
\end{minipage}
\end{center}

\begin{displaymath}
f_{puissance}(x) = \mathds{1}_{[a,+\infty [}(x) \text{ } x^{-\alpha} \times \dfrac{\alpha - 1}{a^{1-\alpha}}
\end{displaymath}

\begin{center}
$\Updownarrow$
\end{center}

\begin{displaymath}
\forall t \in \mathbb{R}, F_{X_{puissance}}(t) = \mathbb{P}_{X_{puissance}}(]-\infty: t]) = \int_{-\infty}^{t} f_{puissance}(x) dx
\end{displaymath}

\begin{displaymath}
= \int_{-\infty}^{t} \mathds{1}_{[a,+\infty [}(x) \text{ } x^{-\alpha} \times \dfrac{\alpha - 1}{a^{1-\alpha}} dx
\end{displaymath}

\begin{displaymath}
= \mathds{1}_{[a,+\infty [}(t) \int_{a}^{t} \text{ } x^{-\alpha} \times \dfrac{\alpha - 1}{a^{1-\alpha}} dx = \mathds{1}_{[a,+\infty [}(t) \dfrac{\alpha - 1}{a^{1-\alpha}} \int_{a}^{t} \text{ } x^{-\alpha} dx
\end{displaymath}

\begin{displaymath}
= \mathds{1}_{[a,+\infty [}(t) \times \dfrac{\overbrace{\alpha - 1}^{= -(1 - \alpha)}}{a^{1-\alpha}} \times \left[ \dfrac{x^{1-\alpha}}{1-\alpha}\right]^{t}_{a} = - \mathds{1}_{[a,+\infty [}(t) \times \left[ \dfrac{x^{1-\alpha}}{a^{1-\alpha}}\right]^{t}_{a}
\end{displaymath}


\begin{displaymath}
= \mathds{1}_{[a,+\infty [}(t) \times \left[ \dfrac{x^{1-\alpha}}{a^{1-\alpha}}\right]^{a}_{t} = \mathds{1}_{[a,+\infty [}(t) \times \underbrace{\left[ x^{1-\alpha} \right]^{a}_{t}}_{= a^{1- \alpha} - t^{1- \alpha}} \times a^{\alpha - 1}
\end{displaymath}

\begin{displaymath}
= \mathds{1}_{[a,+\infty [}(t) \times \left(1- \left(\dfrac{t}{a}\right)^{1-\alpha} \right)
\end{displaymath}


\begin{center}
\begin{minipage}[r]{0.1\textwidth}

\end{minipage}
\begin{minipage}[r]{0.8\textwidth}
Posons $F_{X_{puissance}}^{\phantom{aa}-}$, la fonction définit de la sorte : $\forall y \in ]0,1[$ :

$F_{X_{puissance}}^{\phantom{aa}-}(y)$ = inf $\bigl\{ x \in \mathbb{R} \text{ | } y \leq \mathds{1}_{[a,+\infty [}(x) \times \left(1- \left(\dfrac{x}{a}\right)^{1-\alpha} \right) \bigr\}$
\end{minipage}
\end{center}



\begin{center}
\begin{minipage}[r]{0.1\textwidth}

\end{minipage}
\begin{minipage}[r]{0.8\textwidth}
Or $y \in ]0,1[ \iff 0 < y \iff \bigl\{ x \in \mathbb{R} \text{ | } y \leq \mathds{1}_{[a,+\infty [}(x) \times \left(1- \left(\dfrac{x}{a}\right)^{1-\alpha} \right) \bigr\}$ = $\bigl\{ x \in [a,+\infty [ \text{ | } y \leq \left(1- \left(\dfrac{x}{a}\right)^{1-\alpha} \right) \bigr\}$
\end{minipage}
\end{center}


\newpage

\begin{center}
\begin{minipage}[r]{0.1\textwidth}

\end{minipage}
\begin{minipage}[r]{0.8\textwidth}
inf$\bigl\{ x \in [a,+\infty [ \text{ | } y \leq 1- \left(\dfrac{x}{a}\right)^{1-\alpha} \bigr\}$ = inf$\bigl\{ x \in [a,+\infty [ \text{ | } y - 1 \leq - \left(\dfrac{x}{a}\right)^{1-\alpha} \bigr\}$ = sup$\bigl\{ x \in [a,+\infty [ \text{ | } \left(\dfrac{x}{a}\right)^{1-\alpha} \leq 1-y \bigr\}$ = sup$\bigl\{ x \in [a,+\infty [ \text{ | } \left(\dfrac{x}{a}\right) \leq \sqrt[1-\alpha]{1-y} \bigr\}$ = sup$\bigl\{ x \in [a,+\infty [ \text{ | } x \leq a\sqrt[1-\alpha]{1-y} \bigr\}$

\end{minipage}

Or $a\sqrt[1-\alpha]{1-y} = \dfrac{a}{\underbrace{\sqrt[\alpha-1]{1-y}}_{\in ]0,1[}} \in [a,+\infty [ \iff$

$F_{X_{puissance}}^{\phantom{aa}-}(y) = sup\bigl\{ x \in [a,+\infty [ \text{ | } x \leq a\sqrt[1-\alpha]{1-y} \bigr\}$ = $sup[a,a\sqrt[1-\alpha]{1-y} ]$ = \colorbox{black!30}{$a\sqrt[1-\alpha]{1-y}$}
\end{center}


\begin{center}
\begin{minipage}[r]{0.1\textwidth}

\end{minipage}
\begin{minipage}[r]{0.8\textwidth}
A l'aune de cette information nouvelle, nous pouvons établir notre modèle de la sorte :  
\end{minipage}

\colorbox{black!30}{Soit $U \sim \mathcal{U}(]0,1[)$, $F_{X_{puissance}}^{\phantom{aa}-}(U) \sim X_{puissance}$}
\end{center}


\subsection{Conditions météorologiques et chaines de Markov $(H_k)_{k \in \mathbb{N}^{*}}$}

\begin{center}
\begin{minipage}[r]{0.1\textwidth}

\end{minipage}
\begin{minipage}[r]{0.8\textwidth}
L'on cherche a réaliser un modèle simulant des observations journalières de nos états météorologique.

Pour ce faire (et de façon assez naturelle, nous établirons une \textit{chaine de Markov} $(H_k)_{k \in \mathbb{N}^{*}}$, possédant les propriétés suivantes :

\end{minipage}
\end{center}





\begin{description}

\item[\bsc{* Processus Aléatoire $(H_k)_{k \in \mathbb{N}^{*}}$ : }]

\textit{Soit $(H_k)_{k \in \mathbb{N}^{*}} \in (E, \mathcal{A}, \mathbb{P})^{\mathbb{N}^{*}}$}

\phantom{aaaaaaa} — \textit{On note} $E = \{ \text{beau temps}, \text{temps couvert}, \text{pluie} \}$, dit \textit{« ensemble des états de $(H_k)_{k \in \mathbb{N}^{*}}$ »}.

\phantom{aaaaaaa} — \textit{On note} $\mu_0$, \textit{une mesure de probabilité sur $(E, \mathcal{A})$, dite « loi initiale de $(H_k)_{k \in \mathbb{N}^{*}}$ », tel que $(\mu_0(i))_{i \in [|1,3|]}$, est une permutation quelconque de $(1,0,0)$}.

\phantom{aaaaaaa} — \textit{On note} $Q \in M_3(\mathbb{R})$, \textit{une matrice stochastique, dite « matrice de transition de $(H_k)_{k \in \mathbb{N}^{*}}$ », tel que $\forall k \in \mathbb{N}^{*}$\footnotemark[1]} : 

\begin{displaymath}
\begin{bmatrix}
\mathbb{P}(H_{k+1} = 1 | H_k = 1) & \mathbb{P}(H_{k+1} = 2 | H_k = 1) & \mathbb{P}(H_{k+1} = 3 | H_k = 1)\\
\mathbb{P}(H_{k+1} = 1 | H_k = 2) & \mathbb{P}(H_{k+1} = 2 | H_k = 2) & \mathbb{P}(H_{k+1} = 3 | H_k = 2)\\
\mathbb{P}(H_{k+1} = 1 | H_k = 3) & \mathbb{P}(H_{k+1} = 2 | H_k = 3) & \mathbb{P}(H_{k+1} = 3 | H_k = 3)
\end{bmatrix}
\end{displaymath}


\end{description}




\begin{tcolorbox}[colback=black!30,colbacklower=black!20,colframe=black!20,rightrule=1mm,sidebyside]


\begin{displaymath}
\begin{bmatrix}
p_{1,1} & p_{1,2} & p_{1,3}\\
p_{2,1} & p_{2,2} & p_{2,3}\\
p_{3,1} & p_{3,2} & p_{3,3}
\end{bmatrix}
\end{displaymath}


\tcblower
\begin{tikzpicture}[->,>=stealth',shorten >=1.4pt,auto,node distance=2cm,
                    thick,main node/.style={circle,draw,font=\sffamily\Large\bfseries}]

\node[main node] (1) {1};
\node[main node] (2) [below left of=1] {2};
\node[main node] (3) [below right of=1] {3};

  \path[every node/.style={font=\sffamily\small}]
    (1) edge node [left] {$p_{1,3}$} (3)
        edge [bend right] node[left] {$p_{1,2}$} (2)
        edge [loop above] node {$p_{1,1}$} (1)
    (2) edge node [right] {$p_{2,1}$} (1)
        edge node {$p_{2,3}$} (3)
        edge [loop left] node {$p_{2,2}$} (2)
        %edge [bend right] node[left] {$p_{2,1}$} (1)
    (3) edge node [bend right] {$p_{3,2}$} (2)
        edge [bend right] node[right] {$p_{3,1}$} (1)
        edge [loop right] node[right] {$p_{3,3}$} (3);
\end{tikzpicture}

\end{tcolorbox}



\begin{center}
\begin{minipage}[r]{0.1\textwidth}

\end{minipage}
\begin{minipage}[r]{0.8\textwidth}
Vous trouverez en annexe la construction de notre \textit{« Table de Walker »}.

\end{minipage}
\end{center}


\footnotetext[1]{Pour des raisons évidentes de lisibilité nous confondrons les états \textit{« beau temps »}, \textit{« temps couvert »} et \textit{« pluie »} avec les états respectifs : $1$, $2$, et $3$.}


\newpage
\subsection{Occurrences des sinistres (modèle A)}
\begin{center}
\begin{minipage}[r]{0.1\textwidth}

\end{minipage}
\begin{minipage}[r]{0.8\textwidth}
A chaque états de notre \textit{chaine de Markov} ($(1,2,3)$) est associée une valeur $\lambda_k \in \mathbb{R}_{+}^{*}$ (respectivement $(\lambda_1,\lambda_2,\lambda_3)$.

Notons $N \in \mathbb{N}$, la taille de notre portefeuille.

\textit{« On suppose que, pour chaque jour $k \in \mathbb{N}$, les dates des sinistres déclarés par l'assuré le jour $k$ suivent, conditionnellement à la valeur de $H_k$, un processus de Poisson d'intensité $\lambda_{H_k}$ »}.

\phantom{aaaaaaa}

Ainsi, nous définissons : 

\end{minipage}
\end{center}


\begin{description}

\item[\bsc{* Fonction}]

\textit{$\Delta(\omega) : $ : $\mathbb{R}_{+} \to \mathbb{R}_{+}$}

\begin{displaymath}
	\Delta : 
	\begin{cases}
		\text{$\mathbb{R}_{+} \to \mathbb{R}_{+}$}\\
		t \longmapsto \mathds{1}_{[0,365]}(t) \lambda_{H_{\lfloor t \rfloor}}
	\end{cases}
	.
\end{displaymath}


\item[\bsc{* Fonction}]

\textit{$\mu(\omega) : $ : $\mathbb{R}_{+} \to \mathbb{R}_{+}$}

\begin{displaymath}
\mu(t) = \int_{0}^{t} \Delta(x) dx = \int_{0}^{t} \mathds{1}_{[0,365]}(x) \lambda_{H_{\lfloor x \rfloor}} dx
\end{displaymath}


\item[\bsc{* Suite de processus de poisson in-homogène mélange : }]

\textit{$(R_t^{(k)})_{k \in [|1,N|]}$}

\phantom{aaaaaaa} — \textit{$(R_t^{(k)})_{k \in [|1,N|]}$, une suite de processus de poisson in-homogène mélange indépendants.}

\phantom{aaaaaaa} — \textit{$\forall (k,t) \in [|1,N|] \times \mathbb{R}_{+}$,  $R_t^{(k)} \sim P(\mu(t))$, le  processus de poisson in-homogène mélange indépendants représentant les temps d'occurrence des sinistres du contrat $k$.}

\begin{displaymath}
\forall t \in \mathbb{R}_{+} \text{, } R_t = \sum_{i=1}^{N} R_t^{(i)}
\end{displaymath}

\end{description}


\begin{displaymath}
\forall t \in [0,365] \text{, } \mu(t) = \int_{0}^{t} \Delta(x) dx = \int_{0}^{t} \mathds{1}_{[0,365]}(x) \lambda_{H_{\lfloor x \rfloor}} dx = \int_{0}^{t} \lambda_{H_{\lfloor x \rfloor}} dx
\end{displaymath}

\begin{displaymath}
= \int_{0}^{1} \lambda_{H_{\lfloor x \rfloor}} dx + \int_{1}^{2} \lambda_{H_{\lfloor x \rfloor}} dx + ... + \int_{\lfloor t \rfloor - 1}^{\lfloor t \rfloor} \lambda_{H_{\lfloor x \rfloor}} dx + \int_{\lfloor t \rfloor}^{t} \lambda_{H_{\lfloor x \rfloor}} dx
\end{displaymath}

\begin{displaymath}
= \int_{0}^{1} \lambda_{H_{0}} dx + \int_{1}^{2} \lambda_{H_{1}} dx + ... + \int_{\lfloor t \rfloor - 1}^{\lfloor t \rfloor} \lambda_{H_{\lfloor t \rfloor - 1}} dx + \int_{\lfloor t \rfloor}^{t} \lambda_{H_{\lfloor t \rfloor}} dx
\end{displaymath}

\begin{displaymath}
= \lambda_{H_{0}} + \lambda_{H_{1}} + ... + \lambda_{H_{\lfloor t \rfloor - 1}} + \int_{\lfloor t \rfloor}^{t} \lambda_{H_{\lfloor t \rfloor}} dx
\end{displaymath}


\begin{displaymath}
= \sum_{i=0}^{\lfloor t \rfloor - 1} \lambda_{H_{i}} + \lambda_{H_{\lfloor t \rfloor}} \times (t - \lfloor t \rfloor)
\end{displaymath}


\newpage
\begin{displaymath}
\forall t \in \mathbb{R}_{+} \text{, } R_t = \sum_{i=1}^{N} \underbrace{R_t^{(i)}}_{\sim P(\mu(t))} \sim P(N\times \mu(t)) \iff
\end{displaymath}

\begin{displaymath}
R_t \text{est un processus de poisson in-homogène mélange d'intensité : } \Delta^{*} = N \Delta
\end{displaymath}


\begin{center}
\begin{minipage}[r]{0.1\textwidth}

\end{minipage}
\begin{minipage}[r]{0.8\textwidth}
Il nous suffit donc de simuler une seule loi de poisson pour nos $N$ contrats.
\end{minipage}

Posons donc : $\forall t \in \mathbb{R}$, \colorbox{black!30}{$\Delta^{*}(t) = N \mathds{1}_{[0,365]}(t) \lambda_{H_{\lfloor t \rfloor}}$.}
\end{center}


\begin{center}
\begin{minipage}[r]{0.1\textwidth}

\end{minipage}
\begin{minipage}[r]{0.8\textwidth}
Cette définition a le bon gout d'être aisément majorable par une formule simple : $N \times max(\lambda_1,\lambda_2,\lambda_3)$.
\end{minipage}
\end{center}



\newpage
\subsection{Probabilité de Ruine Annuelle (modèle A)}

\begin{description}

\item[\bsc{* Modèle de Réserves : }]

\textit{$\forall (N, u, c) \in \mathbb{N} \times \mathbb{R}^2$, $(T_i)_{i \in \mathbb{N}^{*}}$, une suite de variables aléatoires identiquement distribuées et croissantes.}

\phantom{aaaaaaa} — \textit{On note $N$, la taille de notre portefeuille.}

\phantom{aaaaaaa} — \textit{On note $u$, l'investissement initial et individuel par assurés.}

\phantom{aaaaaaa} — \textit{On note $c$, le taux de prime par assuré et par unité de temps.}

\phantom{aaaaaaa} — \textit{On note $(T_i)_{i \in \mathbb{N}^{*}}$, la suite des dates de sinistres déclarés (de tous les assurés). La suite est par ailleurs ordonnée.}

\phantom{aaaaaaaaaaaaaaa} — \textit{$\forall t \in \mathbb{R}^{+} :$}

\[ R_t = Nu + Nct - \sum_{i=1}^{+ \infty} \mathds{1}_{[0,t]}(T_i) \text{ } X_i \]


\item[\bsc{* Partition $A$ de $[0,365]$ : }]

\textit{$\forall k \in \mathbb{N}$.}

\phantom{aaaaaaa} — \textit{$n = inf \bigl\{ k \in \mathbb{N}^{*} | T_k \in [0,365] \bigr\}$}

\begin{displaymath}
A_k =
\begin{cases}
	[0,T_1[ \text{ , si k = } 0 \\
	[T_k, T_{k+1}[ \text{ , si } k \in [|1, n-1 |] \\
	[T_n,365] \text{ , si } k = n \\
	\emptyset \text{ , sinon }
\end{cases}
\end{displaymath}


\phantom{aaaaaaaaaaaaaaa} — \textit{$(A_k)_{k \in \mathbb{N}}$ forme une partition de $[0,365]$.}
\end{description}



\begin{center}
\begin{minipage}[r]{0.1\textwidth}

\end{minipage}
\begin{minipage}[r]{0.8\textwidth}
Soit : $\forall \omega \in \Omega$, $R_t(\omega)$, est bien définie sur $[0,365]$.
On a donc : 
\end{minipage}
\end{center}

\[ \min_{t \in [0,365]} R_t(\omega) = \min_{k \in [|0,n|]} (\min_{t \in A_k} R_t(\omega)) \]


\begin{center}
\begin{minipage}[r]{0.1\textwidth}

\end{minipage}
\begin{minipage}[r]{0.8\textwidth}
Cette partition ($(A_k)_{k \in \mathbb{N}}$), nous permet d'exprimer tout minimum local ($\min_{t \in A_k} R_t(\omega)$), sous une forme explicite (plus ou moins simple) : 
\end{minipage}
\end{center}

\[ \forall (k,t) \in \mathbb{N} \times A_k, \text{ } R_t(\omega) = Nu + Nct - \sum_{i=1}^{+ \infty} \mathds{1}_{[0,t]}(T_i(\omega)) \text{ } X_i(\omega) \]

\[ = Nu + Nct - \sum_{i=1}^{k} X_i(\omega) \iff \]

\begin{displaymath}
\min_{t \in A_k} R_t(\omega) =
\begin{cases}
	Nu \text{ , si k = } 0 \\
	Nu + NcT_k(\omega) - \sum_{i=1}^{k} X_i(\omega) \text{ , si } k \in [|1, n |]
\end{cases}
\end{displaymath}



\begin{center}
\begin{minipage}[r]{0.1\textwidth}

\end{minipage}
\begin{minipage}[r]{0.8\textwidth}
, car par soucis de réalisme l'on considère (par soucis de réalisme) que $c \in \mathbb{R}_{+}^{*}.$
\end{minipage}
\end{center}




\begin{center}
\begin{minipage}[r]{0.1\textwidth}

\end{minipage}
\begin{minipage}[r]{0.8\textwidth}
Nous avons donc le résultat suivant : 
\end{minipage}
\end{center}

\[ \min_{t \in [0,365]} R_t(\omega) = Nu + \min_{k \in [|1,n|]} (NcT_k(\omega) - \sum_{i=1}^{k} X_i(\omega)) \]

\[ = Nu + \min_{k \in [|1,n|]} (\sum_{i=1}^{k} (\dfrac{NcT_k(\omega)}{k} - X_i(\omega))) \]


\begin{center}
\begin{minipage}[r]{0.1\textwidth}

\end{minipage}
\begin{minipage}[r]{0.8\textwidth}
Ce résultat nous permet de caractériser l'événement suivant :
\end{minipage}
\end{center}


\[ \left( \min_{t \in [0,365]} R_t < 0 \right) = \left( \exists k \in [|1,n|] \text{ , tel que : } \sum_{i=1}^{k} (\dfrac{NcT_k}{k} - X_i) < -Nu \right) \]

\begin{center}
\begin{minipage}[r]{0.1\textwidth}

\end{minipage}
\begin{minipage}[r]{0.8\textwidth}
Il nous suffirait donc de procédé pas a pas.
\end{minipage}
\end{center}

\[ \text{Soit } (V_j)_{j \in \mathbb{N}} \text{, une suite de v.a.i.i.d., tel que } \forall j \in \mathbb{N}, \]


\[ V_j \sim \mathbb{B}(\mathbb{P} \left( \exists k \in [|1,n|] \text{ , tel que : } \sum_{i=1}^{k} (\dfrac{NcT_k}{k} - X_i) < -Nu \right) )\]



\begin{center}
\begin{minipage}[r]{0.1\textwidth}

\end{minipage}
\begin{minipage}[r]{0.8\textwidth}
Par loi des grands nombres nous avons :
\end{minipage}
\end{center}



\[ \dfrac{1}{p} \sum_{j=1}^{p} V_j \underset{p \to +\infty}{\overset{p.s.}{\longrightarrow}} \mathbb{P}\left( \min_{t \in [0,365]} R_t < 0 \right) \]










\newpage
\section{Commentaires sur les techniques de programmation}
\pagecolor{blue!20}



\section{Les différents modèles}
\subsection{Complexité algorithmique}
\subsection{Vérifications d'usage}


\part{Conclusion}
\section{Commentaires sur la pertinence du modèle}
\section{Évolutions possibles du modèle}
\section{Commentaires}
\section{Annexes et contacts}

\subsection*{Annexes et contacts}



\begin{center}
\begin{minipage}[r]{0.4\textwidth}

\end{minipage}
\begin{minipage}[r]{0.4\textwidth}


\end{minipage}
\end{center}






\end{document}